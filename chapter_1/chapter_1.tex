

% \chapter{Understanding inter-subject functional variability}\label{chap:intro_fmri}
% \markright{{~{\rm \ref{chap:intro_fmri}}. Introduction to fMRI}\hfill}{}

% \label{Chapter_1}



% \vspace*{\fill}
% \newthought{In this chapter}%  we introduce functional magnetic resonance imaging (fMRI). We will start by providing some insight into human brain structure and function. Then, we will introduce the principal brain imaging techniques in use nowadays. Different imaging techniques can be used to answer different neuroscientific questions. Functional MRI, due to its good spatial resolution and whole brain coverage is specially well suited to answer questions relating the localization of brain activity for a given task.

% % Before the data acquired through fMRI can be used in statistical analysis it has to go through a  preprocessing pipeline. In the last part of this chapter we detail the different steps of this pipeline, with special emphasis on the general linear model (GLM), a model that allows to extract time-independent activation coefficients from the fMRI time series in event-related designs.  These activation coefficients will form the basis of statistical studies presented in later chapters. 

% % % In the last part of the chapter we focus on fMRI and we relate the different steps from the scanner images to the statistical studies of later chapters.

% % % The objective of cognitive studies relating the brain's structure to its function is the output of time-independent activation maps for a given task. We will present a model, known as general linear model (GLM) .


% % % {\blue
% % % fMRI consist of successive  brain scans, given in intervals ranging from 1 to 4 seconds. However,  Because of the inherent delay of oxygen consumption in the brain, construction of these maps is not straightforward in the case of fMRI for fast event-related designs. We will describe a model known as general linear model (GLM) that .
% % % }

% % \hspace{20pt}
% % \minitoc
% % \vspace*{\fill}

% % \newpage


% \chapter{Functional neuroimaging modalities}


% The human brain has a volume of around $1200 \; cm^3$ and an average weight  of 1.5 kg. It is composed of neurons, glia cells and blood vessels. Glia cells are responsible for the structural and metabolic support of neurons. About $86$ billion neurons~\citep{Azevedo2009} process and transmit information through electrical and chemical signals. The information is transmitted along the neuron by \emph{action potentials}
% (also called \emph{spikes}), that are short-lasting electrical events in which
% the electrical membrane potential of a cell rapidly rises and falls. 

% \begin{marginfigure}[-0.5cm]
% \includegraphics[width=1.0\linewidth]{chapter_1/chapter_1_neuron.pdf}
% \vspace{-20pt}\caption{
% Schematic view of a neuron, in scale $10^{5}:1$. A neuron has a cell body (\emph{soma}), many regions for receiving information from other neural cells (\emph{dentrites}) and often a nerve fiber called \emph{axon}.
% Adapted from http://commons.wikimedia.org/.}\label{fig:chapter_1_neuron}
% \end{marginfigure}


% A neuron (Fig.~\ref{fig:chapter_1_neuron}) has a cell body (called the \emph{soma}), many regions for
% receiving information from other neural cells (called \emph{dendrites}),
% and
% often an \emph{axon} (\emph{nerve fiber}) for transmitting information to
% other
% cells. Neurons communicate with one another via chemical synapses, where the axon terminal of one cell impinges upon another neuron's dendrite, soma or, less commonly, axon. Neurons can have over 1000 dentritic branches, making connections with tens of thousands of other cells. Synapses can be excitatory or inhibitory and either increase or decrease activity in the target neuron. Some neurons also communicate via electrical synapses, which are direct, electrically conductive junctions between cells.

% \begin{marginfigure}[0.5cm]
% \includegraphics[width=1.\linewidth]{chapter_1/Cajal-shadow.jpg}
% \vspace{-10pt}\caption{
% 	Santiago Ramón y Cajal (Navarre, Spain 1852 – Madrid, Spain 1934) is widely regarded as the father of modern neuroscience. Cajal and italian anatomist Camillo Golgi impersonated the dispute between neuron and reticular theory at the turn of the 20th century. They received a joint Nobel Prize in Physiology and Medicine in 1906.
% }
% \end{marginfigure}



% The human brain can be decomposed in two parts: the \emph{white matter}, constituted by the nerve
% fibers, and the \emph{gray matter} constituted by the neural cell bodies.
% The surface of the human brain is a highly circonvoluted 6-layered structure
% called \emph{neocortex} (or more simply \emph{cerebral cortex}). This layer is folded in a way that increases the amount of surface that can fit into the volume available. A cortical fold is called \emph{sulcus}, and the area between two \emph{sulci} is called a \emph{gyrus}.


% The human cortex is often divided into four ``lobes'', called the frontal lobe, parietal lobe, temporal lobe and occipital lobe (see Figure~\ref{fig:chapter_1_functions}). The left and right side hemispheres of the cortex are broadly similar in shape, and most cortical areas are replicated on both sides. Some areas, however, show strong lateralization, such as areas that are involved in language, located in the vast majority of subject in the left hemisphere.

% % \begin{marginfigure}
% % \includegraphics[width=1.\linewidth]{figures/640px-Phrenology-journal.jpg}
% % \caption{Front page of the Americal Phrenological Journal, 1848. Although now considered a pseudoscience, phrenological thinking has been influential in 19th-century psychiatry and modern neuroscience}
% % \end{marginfigure}

% How the different anatomical structures of the brain correspond to the neural substrate of cognitive functions is one of the oldest debates in neuroscience, defining an entire field: cognitive neuroscience. The idea of linking a given cognitive function to a specific brain region can be traced back to the work of nineteen century phrenologists, who based their localizationist attempts on the shape of the skull. In the 20th century, a group of neuropsychologists, in absence of direct means to investigate brain activity, studied patients with cortical damages observing that some focal lesions were associated with relatively global effects on behavior. This lead them to argue against a strictly localizationist view of brain organization. Nowadays it is widely recognized that the activity of specific brain regions underlie many cognitive functions (e.g.vision, in occipital areas). At the same time, the relevance of \emph{brain networks} encompassing different anatomical regions for the multimodal integration of features necessary for higher level cognitive functions (e.g.attention in the fronto-parietal network) ~\citep{gazzaniga2004cognitive} has been acknowledged.


% \begin{figure}
% \begin{center}
% \center \includegraphics[width=0.6\linewidth]{chapter_1/chapter_1_lobe.pdf}
% \center \includegraphics[width=1.\linewidth]{chapter_1/chapter_1_functions.pdf}
% \end{center}
% \caption{Lobes and some functional regions
% of the human brain (left hemisphere).
% Within each lobe are numerous cortical areas, each associated with a particular function such as
% sensory areas (\emph{e.g. visual cortex, auditory cortex}) that receive and
% process information from sensory organs, motors areas (\emph{e.g. primary motor
% cortex, premotor cortex}) that control the movements of the subject,
% and associative areas (\emph{e.g. Broca's area, Lateral Occipital Complex -- LOC
% --
% or Intra Parietal Sulcus -- IPS --}) that process the high-level information
% related to cognition. The experiments detailed in this thesis are related to
% object recognition (\emph{visual cortex} and \emph{LOC}) and number processing
% (\emph{parietal cortex} and \emph{IPS}). Source: adapted from \citep{michel2010understanding}.}\label{fig:chapter_1_functions}. 
% \end{figure}


\chapter{Brain networks and functional connectivity}
\section{Resting-state networks}
\section{Clinical implications}
\section{The functional brain as dynamic graph}
% \section{Task activation as a small perturbation of resting-state networks}

\chapter{Inter-subject variability in the human brain}
\section{Structural variability: the human brain varies in structure, size, and shape}
\section{Functional variability: the human brain varies in function}

\chapter{EPI-to-EPI inter-subject nonlinear registration}\label{Chapter_1_Section_1}
\section{Related works}
\section{Proposed method}


% Until the advent in the 1920s of non-invasive neuroimaging modalities, most of the accumulated knowledge of the brain came from the study of lesions, post-mortem analysis and invasive experimentations. With the advent of modern, non-invasive imaging techniques, several aspects of the human brain are revealed in vivo with high degree of precision.

% Several brain imaging techniques are available today. These can be divided into \emph{structural} or \emph{anatomical} and \emph{functional} imaging techniques. While structural imaging provides details on morphology and structure of tissues, functional imaging reveals physiological activities such as changes in metabolism, blood flow, regional chemical composition, and absorption. In this section we will discuss briefly the main functional neuroimaging modalities available today.

% \begin{itemize}
% \item{\bf {Electroencephalography - EEG}}
% % \begin{marginfigure}[3cm]
% % \center \includegraphics[width=.8\linewidth]{figures/eeg2.jpg}
% % \caption{EEG Cap}
% % \end{marginfigure}
% is a widely used modality for functional brain
% imaging. \emph{EEG} measures electrical activity along the scalp. EEG activity  reflects the synchronous activity of a population of neurons that have similar spatial orientation. If the cells do not have similar spatial orientation, their ions do not line up and thus do not create detectable waves. Pyramidal neurons of the cortex are thought to produce most of the EEG signals because they are well-aligned and fire together. Because voltage fields fall off with the square of distance, activity from deep sources is more difficult to detect than currents near the skull. Due to the ill-posed problem of
% volumetric data reconstruction from surface measurements,
% \emph{EEG} has a poor spatial resolution compared to other modalities
% such as \emph{fMRI}.

% \item{\bf {Stereotactic electroencephalography - sEEG}} is an invasive version of
% \emph{EEG}, based on intra-cranial recording. It measures the electrical
% currents
% within some regions of the brain using deeply implanted electrodes, localized
% with a stereotactic technique.
% This approach has the good temporal resolution of \emph{EEG} and enjoys an
% excellent spatial resolution. However, \emph{sEEG}
% is very invasive and is only performed for medical purpose (\emph{e.g}
% localization of epilepsy foci) and has a limited coverage (only the regions with electrodes).
% A close approach is \emph{Electrocorticography -- ECog --} that uses
% electrodes placed directly on the exposed surface of the brain. Even in this case its usage is restricted to medical purposes.


% \begin{marginfigure}[5cm]
% \center \includegraphics[width=1.\linewidth]{chapter_1/meg.pdf}
% \caption{Magnetic field measured with MEG on a somato-sensory experiment. It is a 2D topography 20 ms after stimulation. Source: \citep{gramfort:09}}
% \end{marginfigure}
% \item{\bf{Magnetoencephalography - MEG}}
% measures the magnetic field induced by neural electrical activity.
% The synchronized currents in neurons create magnetic fields of a
% few hundreds of femto Tesla ($fT$) that can be detected using specific devices. 
% Although EEG and MEG signals originate from the same neurophysiological processes, there are important differences. Magnetic fields are less distorted than electric fields by the skull and scalp, which results in a better spatial resolution of the MEG. Whereas EEG is sensitive to both tangential and radial components of a current source in a spherical volume conductor, MEG detects only its tangential components. Because of this EEG can detect activity both in the sulci and at the top of the cortical gyri, whereas MEG is most sensitive to activity originating in sulci. EEG is, therefore, sensitive to activity in more brain areas, but activity that is visible in MEG can be localized with more accuracy. Note that EEG and MEG can be measured simultaneously.


% \item{\bf{Positron emission tomography - PET}}
%  is an imaging modality based on the
% detection of a radioactive tracers introduced in the body of the subject. The
% tracers (or \emph{radionuclide} decay) emit a positron which can in turn emit,
% after recombination with an electron, a pair of photons that are detected
% simultaneously. PET therefore provides a quantitative measurement of the physiological activity. It can also be used for functional imaging, by choosing a specific tracer.
% In particular, the \emph{fluorodeoxyglucose} (or \emph{FDG}), is used for
% imaging the metabolic activity of a tissue. This is based on the assumption that areas of high radioactivity are associated with brain activity.
% \emph{PET} has two major limitations: the tracers required for
% \emph{PET} are produced by cyclotrons (a type of particle accelerator),
% which implies an heavy logistic. Furthermore, the use of radio-tracers is not harmless
% for the
% health of the subjects so \emph{PET} is now used for medical purpose only.

% \begin{marginfigure}[0cm]
% \center \includegraphics[width=.8\linewidth]{chapter_1/212px-PET-image.jpg}
% \caption{PET scan of a human brain. 
% PET measures indirectly the flow of blood to different parts of the brain, which is, in general, believed to be correlated with neural activity.
% Souce: wikipedia.org}
% \end{marginfigure}


% \item{\bf{Single photon emission computed tomography - SPECT}} is an imaging modality based on the detection of a radioactive tracer. SPECT is similar
% to \emph{PET} in its use of radioactive tracer material. However, the
% measure in \emph{SPECT} is the direct consequence of the tracer (the tracer
% emits gamma radiation), where \emph{PET} is based on an indirect consequence of
% the tracer (positron then gamma radiation). The spatial resolution is slightly worse
% than \emph{PET}.
% %
% \emph{SPECT} can be used for functional brain imaging, by using a specific
% tracer which will be assimilated by the tissue in an amount proportional to
% the cerebral blood flow.


% \item{\bf{Near-infrared spectroscopy - nIRS}}
%  is a recent modality for
% medical imaging. \emph{nIRS} is based on the fact that the absorption of the
% light in the
% near-infrared domain contains information on the blood flow and blood
% oxygenation level. It is non-ionizing (harmless), and the instruments are
% not too expensive. However, the spectra obtained by \emph{nIRS} can be difficult
% to interpret, and this technique, which requires a complex calibration, measures
% signals only close to the outer layer of the cortex.

% \newglossaryentry{fMRI}{name=fMRI,description=Functional Magnetic Resonance Imaging}

% \item{\bf{Functional MRI} -- {\gls{fMRI}}} is
% a widely used method for functional brain imaging, because it is
% non-invasive, has a good spatial
% resolution ($1mm^3$), and provides access,
% albeit indirectly, to the neural activity.
% Moreover, in standard acquisitions, \emph{fMRI} yields a
% full-brain coverage, as it does not
% restrict the study to superficial layers or predefined regions of the cortex.

% \end{itemize}


% Different modalities have different trade offs in terms of spatial and temporal resolution. For example, EEG and MEG enjoy temporal resolutions of the order of few miliseconds and are thus well suited for studies of temporal dynamics of information processing but have limited spatial resolution. On the other hand, fMRI enjoys a better spatial resolution but the temporal resolution is around 1 second. Furthermore, as we will see in the next section, temporal resolution in fMRI is further limited by the slow spread of hemodynamic response, which lasts around 20 seconds after the stimuli presentation.




% \begin{figure}[h!tb]
% \center \includegraphics[width=0.8\linewidth]{chapter_1/chapter_1_methods.pdf}
% \caption[Spatial and temporal resolutions of the different modalities commonly
% used for functional imaging]{Spatial and temporal resolutions of different
% modalities commonly
% used for functional imaging. A typical fMRI acquisition (as of 2014) enjoys spatial resolution of the order of {$1-3mm^3$} and temporal resolution of the order of 1-3 seconds.}\label{fig:chapter1_methods}
% \end{figure}

% Certain imaging techniques are more adapted than other to answer certain neuroscientific questions. Due to its good spatial resolution and whole brain coverage, fMRI is particularly well adapted to \emph{localize} the effect of a certain experimental condition. This task is not reduced to the construction of brain maps, but also involves the understanding of the underlying brain connectivity~\citep{johansen2005functional,behrens2006consistent} and the effects regions exert on each other in a certain experimental context~\citep{pessiglione2007brain, behrens2007learning}. One of the main hopes in functional imagining is that it might be used as an objective diagnosis tool for several diseases. In particular, the aim is to find some \emph{biomarkers} for psychiatric diseases by comparing different population of patients: this is the case for autism, schizophrenia or Alzheimer's disease. 


\begin{fullwidth}
\bibliographystyle{plainnat}
\bibliography{chapter_1/biblio1}
\end{fullwidth}





