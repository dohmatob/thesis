\chapter{\;\;Concluding remarks}\label{chap:conclusion}
\markright{{~{\rm \ref{chap:conclusion}}. Conclusion and Perspectives}\hfill}{}

\minitoc

\section{Summary of main contributions}
This thesis was kicked-off with the goal of developing methods for modeling inter-subject
functional variability, the aim being to enhance the estimation of functional connectomes
(data-driven regions of interest, connectivity matrices, etc.)
across populations of subjects. The quest led to the study of, and proposal of methods for, structured (sparsity, smoothness, etc.) multi-variate models for brain encoding / decoding~\citep{dohmatob2015speeding,abrahamregion,eickenberg2015total,pelle2016multivariate}. Nonlinear registration of functional brain images also came up as a natural concern, and we contributed a method for direct registration of functional brain images~\citep{dohmatob2016epi2epi} (submitted to \textit{Neuroimage} journal).

We also improved the current state-of-the-art in ROI extraction and dimensionality reduction by combining techniques from online learning and structured sparsity (like TV-L1) to propose a novel scalable dictionary-learning framework for obtaining decompositions of brain images, which are closer to known neuro-biological organization of the brain: networks made of spatially localized smooth components with sharp boundaries.

The ultimate indicator for having understood a phenomenon is being able to recreate it, at least approximately. Indeed, Feynman once said, \textit{``What I cannot (re)create, I do not understand!''} By combing techniques in generative modeling and ensembles, we improved state-of-the-art methods for predicting task-based activation maps (at the individual level!) from resting-state fMRI data, with accuracy well above chance.
% For example, across-subject median $R^2$ score on test data is close to $0.2$ for the different tasks.
This work is being prepared for journal submission.

Due to the intimate relationship between modeling and optimization, the
bulk of this work was made possible by development of new or improvement of existing methods of
optimization, with scalability and robustness at heart~\citep{dohmatob2014benchmarking,dohmatob2015local,varoquaux2015faasta,dohmatob2015simple}.

Some of the work done in the thesis have cross-fertilized other collaborative papers like~\citep{rahim2015integrating,thirion2014fmri}.


\section{Ongoing work and future directions}
\paragraph{Unified view on structured models for brain data.}
We are preparing journal paper synthesizing all our contributions in the regarding structured models for brain decoding and segmentation presented in chapter \ref{chap:structured_priors}. This will bring these methods to the doorsteps of the neuroscience practitioner.
\paragraph{Proximal dictionary updates for structured online dictionary-learning.}
The ideas presented in chapter \ref{chap:proxdict} have a great potential to extend the classical dictionary-learning technology providing the practitioner with a modeling framework incorporating
a much large class of constraints --namely proximable penalty functions-- than is currently being done. Though the convergence of the so-obtained methods seem to directly follow from existing literature on online coordinate-descent methods, a more careful theoretical treatment of the framework is still to be done.
\paragraph{Non-linear generative models for inter-subject brain data and prediction of task-fMRI activity from resting-state data.}
As concerns the modelling of inter-subject variability (chapters \ref{chap:func_var} and \ref{chap:rsfmri2tfmri}), most of the work done in this thesis can be cast in a more flexible framework of generative encoder-less models (see Fig. \ref{fig:gen_net}, for example)\footnote{Since the encoding representation is gotten by simply minimizing a reconstruction loss between the generated and the true brain image.}, with the space of parameters carefully constrained to ensure tractability and intepretability.
  
% \clearpage
\bibliographystyle{plainnat}
\bibliography{bib_all}
